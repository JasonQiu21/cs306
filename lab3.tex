\documentclass[a4paper,10pt]{article}
\usepackage[utf8]{inputenc}
\usepackage{amssymb}
\usepackage{amsfonts}
\usepackage{amsmath}
\usepackage{enumerate}
\setlength{\parindent}{0pt}
\usepackage[margin=1in]{geometry}

\usepackage{listings}
\usepackage{color}

\definecolor{dkgreen}{rgb}{0,0.6,0}
\definecolor{gray}{rgb}{0.5,0.5,0.5}
\definecolor{mauve}{rgb}{0.58,0,0.82}

\lstset{frame=tb,
  language=Python,
  aboveskip=3mm,
  belowskip=3mm,
  showstringspaces=false,
  columns=flexible,
  basicstyle={\small\ttfamily},
  numbers=none,
  numberstyle=\tiny\color{gray},
  keywordstyle=\color{blue},
  commentstyle=\color{dkgreen},
  stringstyle=\color{mauve},
  breaklines=true,
  breakatwhitespace=true,
  tabsize=3
}
\begin{document}

Jason Qiu

CS 306 Lab $\#3$ Writeup

\emph{I pledge my honor that I have abided by the Stevens Honor System.}

In Lab 2, we discussed methods for computing modular exponentiation. The most naive method is to simply iteratively multiply:

$$a^b \mod N = (a \mod N) \cdot (a^{b-1} \mod N) \mod N$$

This approach will iterate equal to the value of the exponent $b$, so the algorithm is in $\mathcal{O}(n)$. A slightly better method is a divide-and-conquer method for computing modular exponentiation. The process is as follows:
\begin{equation*}
a^b \mod N = 
\begin{cases}
	a \mod N, & a = 1\\
	(a^{b/2} \mod N) \cdot (a^{b/2} \mod N) \mod N, & a = 0 \mod 2\\
	(a \mod N)(a^{b/2} \mod N) \cdot (a^{b/2} \mod N) \mod N, & a = 1 \mod 2
\end{cases}
\end{equation*}

Here is an implementation of these two methods in Python:

\begin{lstlisting}
#lab2.py

def bad_pow(a,b,m):
    """compute a^b mod m"""
    result = 1
    for i in range(b):
        result = (result * a)%m
    return result

memo = {}

def fast_pow(a, b, m):
    """Compute a^b mod m"""
    if (a, b, m) in memo:
        return memo[(a, b, m)]
    elif b == 1:
        memo[(a, b, m)] = a%m
        return memo[(a, b, m)]
    elif b%2:
        memo[(a, b, m)] = (a%m * fast_pow(a, b//2, m) * fast_pow(a, b//2, m))%m
        return memo[(a, b, m)]
    else:
        memo[(a, b, m)] = (fast_pow(a, b//2, m) * fast_pow(a, b//2, m))%m
        return memo[(a, b, m)]

# Version with no memoization
def no_memo_pow(a, b, m):
    """Compute a^b mod m"""
    if b == 1:
        return a%m
    elif b%2:
        return (a%m * no_memo_pow(a, b//2, m) * no_memo_pow(a, b//2, m))%m
    else:
        return (no_memo_pow(a, b//2, m) * no_memo_pow(a, b//2, m))%m

if __name__ == '__main__':
	# Prints just to test if functions work
    print(no_memo_pow(20, 2001, 33))
    print(fast_pow(20, 2001, 33))
    print(pow(20, 2001, 33))
\end{lstlisting}

Because each recursive step reduces the exponent by half, this algorithm is bounded by $\mathcal{O}(\log_2(n))$. However, each step produces about double the recursive calls than the previous, so without memoization the function is actually much slower.
\newpage
In Lab 3, we discussed a more efficient precomputation algorithm. The exponent of $a^b \mod N$ can be rewritten as the $i$-bit binary number $\sum_{0}^{i-1}b_i2^i$, where $b_i$ is the $i$-th digit. Then, the entire exponent can be written as follows:
$$ a^b = a^{\sum_{0}^{i-1}b_i2^i} = \prod_{0}^{i-1} a^{b_i2^i} \mod N$$
Using the fact that $a^{2^i} = a^{2 \cdot 2^{i-1}} = (a^{2^{i-1}})^2 =  a^{2^{i-1}} \cdot a^{2^{i-1}}$, and the fact that we only care about the final result modulo $N$, we can compute each $a^i$ using the previous quickly by squaring the previous base modulo $N$. The corresponding Python code is as follows:
\begin{lstlisting}
#lab3.py
from lab2 import no_memo_pow, fast_pow, bad_pow # Divide-and-conquer algorithm for modular exponentiation (implemented last week)
from random import randint
from time import time_ns

def faster_pow(a, b, m):
    """Compute a^b mod m"""
    result = 1
    prev = None
    x_i = a%m # x_0 = a%m
    for i in range(len(bin(b)) - 2): # Because bin returns string '0b{binary number}', we subtract the two characters from it
        if bool((b&(2**i))): # Get the ith bit from left
            result *= x_i
        prev = x_i
        x_i = (prev**2)%m # In general, x_i = x_{i-1}^2 (mod m)
    return result%m

if __name__ == '__main__':
    N = 100003
    # Generate two random ints such that 2^31 <= a,b < 2^32
    a = randint(2**31, 2**32-1)
    b = randint(2**31, 2**32-1)

    print(f"a = {a}\nb = {b}")

    for k, v in {'built-in':pow, 'precomputation': faster_pow, 'memoized divide-and-conque': fast_pow, 'divide-and-conquer': no_memo_pow}.items():
        print(f"Using {k} algorithm for modular exponentiation.", end="")
        try:
            start = time_ns()
            built_in = v(a, b, N)
            end = time_ns()
            print(f" Result: {built_in}. Time elapsed: {end-start} ns")
        except RecursionError:
            print(f" Result: recurtionError. Time elapsed: [N/A] ns")
\end{lstlisting}
This algorithm only needs to iterate a number of times equal to $b$'s length, meaning it's bounded by $\mathcal{O}(\log_2(n))$ as well. However, it doesn't need to make redundant calls and is therefore faster than \verb|fast_pow|. \verb|lab3.py| also contains code to compare the speed of each method, alongside the default Python method. Example terminal output:
\lstset{language=bash, keywordstyle=\color{black}}
\begin{lstlisting}
a = 4114569472
b = 3556158952
Using built-in algorithm for modular exponentiation. Result: 12600. Time elapsed: 10262 ns
Using precomputation algorithm for modular exponentiation. Result: 12600. Time elapsed: 45432 ns
Using memoized divide-and-conquer algorithm for modular exponentiation. Result: 12600. Time elapsed: 118223 ns
Using divide-and-conquer algorithm for modular exponentiation. Result: 12600. Time elapsed: 636955999820 ns
Using iterative algorithm for modular exponentiation. Result: 12600. Time elapsed: 295702998969 ns
\end{lstlisting}
Evidently, precomputation is much quicker than the naive (iterative) approach. Interestingly, without memoization, the redundant function calls of the divide-and-conquer approach actually make it slower than the naive approach.
\end{document}
